% Options for packages loaded elsewhere
\PassOptionsToPackage{unicode}{hyperref}
\PassOptionsToPackage{hyphens}{url}
%
\documentclass[
]{article}
\usepackage{lmodern}
\usepackage{amssymb,amsmath}
\usepackage{ifxetex,ifluatex}
\ifnum 0\ifxetex 1\fi\ifluatex 1\fi=0 % if pdftex
  \usepackage[T1]{fontenc}
  \usepackage[utf8]{inputenc}
  \usepackage{textcomp} % provide euro and other symbols
\else % if luatex or xetex
  \usepackage{unicode-math}
  \defaultfontfeatures{Scale=MatchLowercase}
  \defaultfontfeatures[\rmfamily]{Ligatures=TeX,Scale=1}
\fi
% Use upquote if available, for straight quotes in verbatim environments
\IfFileExists{upquote.sty}{\usepackage{upquote}}{}
\IfFileExists{microtype.sty}{% use microtype if available
  \usepackage[]{microtype}
  \UseMicrotypeSet[protrusion]{basicmath} % disable protrusion for tt fonts
}{}
\makeatletter
\@ifundefined{KOMAClassName}{% if non-KOMA class
  \IfFileExists{parskip.sty}{%
    \usepackage{parskip}
  }{% else
    \setlength{\parindent}{0pt}
    \setlength{\parskip}{6pt plus 2pt minus 1pt}}
}{% if KOMA class
  \KOMAoptions{parskip=half}}
\makeatother
\usepackage{xcolor}
\IfFileExists{xurl.sty}{\usepackage{xurl}}{} % add URL line breaks if available
\IfFileExists{bookmark.sty}{\usepackage{bookmark}}{\usepackage{hyperref}}
\hypersetup{
  hidelinks,
  pdfcreator={LaTeX via pandoc}}
\urlstyle{same} % disable monospaced font for URLs
\usepackage{longtable,booktabs}
% Correct order of tables after \paragraph or \subparagraph
\usepackage{etoolbox}
\makeatletter
\patchcmd\longtable{\par}{\if@noskipsec\mbox{}\fi\par}{}{}
\makeatother
% Allow footnotes in longtable head/foot
\IfFileExists{footnotehyper.sty}{\usepackage{footnotehyper}}{\usepackage{footnote}}
\makesavenoteenv{longtable}
\setlength{\emergencystretch}{3em} % prevent overfull lines
\providecommand{\tightlist}{%
  \setlength{\itemsep}{0pt}\setlength{\parskip}{0pt}}
\setcounter{secnumdepth}{-\maxdimen} % remove section numbering

\author{}
\date{}

\begin{document}

\textbf{Pensions \& OPEB Projections Model}

\textbf{{Summary:}} The following document expounds the methodologies
used to calculate the projections of a given city. The city of Boston's
other post-employment benefits (OPEB) fund is used as a case study to
further illustrate these calculations.

\begin{enumerate}
\def\labelenumi{\Roman{enumi}.}
\item
  \begin{quote}
  OPEB Assets Fund
  \end{quote}
\end{enumerate}

\textbf{{Data and Source:}}

\begin{itemize}
\item
  \begin{quote}
  Current Value of Fund: For the city of Boston, it was
  \textbf{\$204,567,000} in 2013.
  \end{quote}
\end{itemize}

\begin{quote}
(\textbf{{Source}}: Boston 2014 CAFR → Notes to the basic Financial
Statement → 12. OPEB → d. Funded Status and Funding Progress of the
Plan)
\end{quote}

\begin{itemize}
\item
  \begin{quote}
  Inflation Rate: For the city of Boston, it was \textbf{2\%} (0.02).
  \end{quote}
\end{itemize}

\begin{quote}
(\textbf{{Source}}: OPEB\_6\_cities.xlsx → Fund Projections → Cell E4)
\end{quote}

\begin{itemize}
\item
  \begin{quote}
  Assets Rate of Return: For the city of Boston, it was \textbf{3\%}
  (0.03).
  \end{quote}
\end{itemize}

\begin{quote}
(\textbf{{Source}}: OPEB\_6\_cities.xlsx → Fund Projections → Cell E3)
\end{quote}

\begin{itemize}
\item
  \begin{quote}
  Employer and Employee Contributions: The city of Boston pledged to
  contribute \textbf{\$40 million} per year to the fund. There were no
  employee contributions.
  \end{quote}
\end{itemize}

\begin{quote}
(\textbf{{Source}}: opeb\_paper\_20160405.pdf → page 6)
\end{quote}

\textbf{{Methodology:}}

The following formula is used to get the value of the fund in the next
period or year:

\(\text{Fun}d_{t + 1\ } = \ Fund_{t}\ *\ (1\  + Assets\ Rate\ of\ Return)\  + \ Employer\ Contribution_{t}\  + \ Employee\ Contribution_{t}\)

The following formulas are used to get the value of the employer and
employee contributions in the next period:

\(\text{Employer\ Contributio}n_{\ t + 1}\  = \ Employer\ Contribution_{t}\ *\ (1\  + \ Inflation\ Rate)\ \)

\(\text{Employee\ Contributio}n_{\ t + 1}\  = \ Employee\ Contribution_{t}\ *\ (1\  + \ Inflation\ Rate)\)

\textbf{{Results:}}

Using the data and the methodology presented above, we will show the
projections of the city of Boston's OPEB Fund from 2014 to 2025 in Table
1.

\begin{longtable}[]{@{}lll@{}}
\toprule
Year & City of Boston Fund Value & Employer Contribution\tabularnewline
\midrule
\endhead
2014 & 250704 & 40000\tabularnewline
2015 & 298225 & 40800\tabularnewline
2016 & 347972 & 41616\tabularnewline
2017 & 400027 & 42448\tabularnewline
2018 & 454476 & 43297\tabularnewline
2019 & 511407 & 44163\tabularnewline
2020 & 570912 & 45046\tabularnewline
2021 & 633085 & 45947\tabularnewline
2022 & 698025 & 46866\tabularnewline
2023 & 765832 & 47803\tabularnewline
2024 & 836610 & 48759\tabularnewline
2025 & 910467 & 49734\tabularnewline
\bottomrule
\end{longtable}

\textbf{Table 1. City of Boston Fund Projections (2014 to 2025)}

II. OPEB Liabilities Projection Model

{\textbf{Data and Source:}}

\begin{itemize}
\item
  \begin{quote}
  Accrued Actuarial Liability (AAL): It is the present value of benefits
  accrued in past years, i.e years of employment prior to the valuation
  date. For the city of Boston it was \textbf{\$2,257,699,000} in 2013.
  \end{quote}
\end{itemize}

\begin{quote}
(\textbf{{Source:}} Boston 2014 CAFR → Notes to the basic Financial
Statement → 12. OPEB → d. Funded Status and Funding Progress of the
Plan)
\end{quote}

\begin{itemize}
\item
  \begin{quote}
  Discount Rate: It refers to the interest rate used to determine the
  present value of future OPEB benefit payments. It plays a pivotal role
  in assessing how much should be contributed now to ensure that an
  adequate level of resources is available in the future. In fact, the
  higher the discount rate, the lower the present value, and vice versa.
  The City of Boston used a discount rate of \textbf{7.5\%} (0.075) in
  2013.
  \end{quote}
\end{itemize}

\begin{quote}
(\textbf{{Source:}} Boston 2014 CAFR → Notes to the basic Financial
Statement → 12. OPEB → e. Actuarial Methods and Assumptions)
\end{quote}

\textbf{{Methodology:}}

The following formula is used to get the OPEB Actuarial Accrued
Liability using a discount different from the one used by the city
officials.

\(\text{AA}L_{\text{new\ rate}}\  = \frac{\text{AA}L_{\text{city\ rate}}\ *\ (1\  + \ city\ rate)^{n}}{(1\  + \ new\ rate)^{n}}\)

Where \(city\ rate,\ new\ rate\ \)and \(n\)are the discount rate used by
the city, the new discount rate and the number of periods (or years)
respectively.

As an illustration, we will evaluate the AAL for the city of Boston
using a more realistic discount rate of 3\% over a period of 15 years.

\(\text{AA}L_{3\%}\  = \frac{2257699000\ \ *\ (1.075)^{15}}{(1.03)^{15}}\  = \$ 4,287,801,000\ \)

III. OPEB Unfunded Actuarial Accrued Liability/ Funded Ratio

The Unfunded Actuarial Accrued Liability (UAAL) is the difference
between the AAL and the present value of the OPEB Asset Fund. Therefore
the UAAL is the amount that is still ``owed'' to the fund for past
obligations.

\end{document}
